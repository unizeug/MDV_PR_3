\newcommand{\institut}{Institut f\"ur Energie und Automatisiertungstechnik}
\newcommand{\fachgebiet}{Elektronische Mess- und Diagnosetechnik}
\newcommand{\veranstaltung}{Praktikum Messdatenverarbeitung}
\newcommand{\pdfautor}{Dirk Babendererde (321 836), Thomas Kapa (325 219)}
\newcommand{\autor}{Dirk Babendererde (321 836)\\ Thomas Kapa (325 219)}
\newcommand{\pdftitle}{Praktikum Messdatenverarbeitung Termin 1}
\newcommand{\prototitle}{Praktikum Messdatenverarbeitung \\ Termin 1}
\newcommand{\aufgabe}{}


\newcommand{\gruppe}{Gruppe: G2 Fr 08-10}
\newcommand{\betreuer}{Betreuer: J\"urgen Funk}



\input{../../../packages/tu_header_8}


% \lstlistoflistings
\definecolor{darkgray}{rgb}{0.95,0.95,0.95}
\lstset{language=Scilab}
\lstset{inputencoding=utf8}
%\lstset{extendedchars=true} % Umlaute an der richtigen stelle und nicht am Anfang ausgeben
\lstset{backgroundcolor=\color{darkgray}}
\lstset{numbers=left, numberstyle=\tiny, stepnumber=1, numbersep=7pt, breaklines=true}
\lstset{keywordstyle=\color{red}\bfseries\emph}
\lstset{
breaklines,
numbers=left,
frame=single,
xleftmargin=-2cm,
xrightmargin=-1.5cm
}
% enables UTF-8 in source code: (dirty, dirty hack)
\lstset{literate=
    %Deutsch
    {ä}{{\"a}}1 {ö}{{\"o}}1 {ü}{{\"u}}1 {Ä}{{\"A}}1 {Ö}
    {{\"O}}1 {Ü}{{\"U}}1 {ß}{\ss}1
    %Türkisch
    {â}{{\^{a}}}1 {Â}{{\^{A}}}1 {ç}{{\c{c}}}1 {Ç}{{\c{C}}}1 {ğ}{{\u{g}}}1 {Ğ}{{\u{G}}}1 {ı}{{\i}}1 {İ}{{\.{I}}}1 {ö}{{\"o}}1 {Ö}{{\"O}}1 {ş}{{\c{s}}}1
    {Ş}{{\c{S}}}1 {ü}{{\"u}}1 {Ü}{{\"U}}1
    %Polish
    {ą}{{\k{a}}}1 {ć}{{\'c}}1 {ę}{{\k{e}}}1 {ł}{{\l{}}}1 {ń}{{\'n}}1 {ó}{{\'o}}1 {ś}{{\'s}}1 {ż}{{\.z}}1 {ź}{{\'z}}1 {Ą}{{\k{A}}}1 {Ć}{{\'C}}1
    {Ę}{{\k{E}}}1 {Ł}{{\L{}}}1 {Ń}{{\'N}}1 {Ó}{{\'O}}1 {Ś}{{\'S}}1 {Ż}{{\.Z}}1 {Ź}{{\'Z}}1
    %Spanish
    {á}{{\'a}}1 {é}{{\'e}}1 {í}{{\'i}}1 {ó}{{\'o}}1 {ú}{{\'u}}1 {ñ}{{\~n}}1
}

%     \lstinputlisting{./praktikum6.sce}



%---------------------------------------------------------------------
%---------------------------------------------------------------------
%---------------------------------------------------------------------

\section{Vorbereitungsaufgaben}


\bq

Eine digitale Messkette besteht aus einem Sensor, der Signalkonditionung (z.B.
Linearisierung, Verstärkung \ldots), dem Antialiasingfilter, einem Abtast- und
Halteglied und einem Analog-Digital-Umsetzer.
Der Analog-Digital-Umsetzer hat eine minimale Abtastfrequenz, z.B. dadurch, dass
man ein Register als Referenz für einen Timer nutzt, dass aber eine begranzte
Anzahl an Bits zur Verfügung stellt.
Sensor, Signalkonitionierung und Antialiasingfilter sitzen in der Wandlerbox,
Abtast- und Haltglied und ADU sitzen auf dem Microcontroller.

\eq

\end{document}
